\documentclass[psamsfonts]{amsart}
\usepackage[utf8]{inputenc}
\usepackage{amsmath,amssymb,amscd,latexsym,enumitem,amsthm,  bbding}
\usepackage{mathrsfs}

\usepackage[hidelinks]{hyperref}
\usepackage[mathscr]{euscript}
\usepackage[all]{xy}
\usepackage{tikz}
\usetikzlibrary{positioning} %Needed for the basic examples
\usetikzlibrary{arrows,decorations.pathmorphing,shapes} %Needed for the crazy example
\usepackage{tikz-cd}

%!--------Theorem Environments--------
%!theoremstyle{plain} --- default
\newtheorem{thm}{Theorem}[section]
\newtheorem{cor}[thm]{Corollary}
\newtheorem{prop}[thm]{Proposition}
\newtheorem{lem}[thm]{Lemma}
\newtheorem{conj}[thm]{Conjecture}
\newtheorem{quest}[thm]{Question}

\theoremstyle{definition}
\newtheorem{defn}[thm]{Definition}
\newtheorem{defns}[thm]{Definitions}
\newtheorem{con}[thm]{Construction}
\newtheorem{exmp}[thm]{Example}
\newtheorem{exmps}[thm]{Examples}
\newtheorem{notn}[thm]{Notation}
\newtheorem{notns}[thm]{Notations}
\newtheorem{addm}[thm]{Addendum}
\newtheorem{exer}[thm]{Exercise}

\theoremstyle{remark}
\newtheorem{rem}[thm]{Remark}
\newtheorem{rems}[thm]{Remarks}
\newtheorem{warn}[thm]{Warning}
\newtheorem{sch}[thm]{Scholium}

\makeatletter
\let\c@equation\c@thm
\makeatother
\numberwithin{equation}{section}



\title{Basic Intoduction to Sheaves and Sheaf Cohomology}
\author{Andrew Doumont}
\date{\today}
\begin{document}

\begin{abstract}
  This is an expository paper outlining some basics of sheaf theory focused primarily on motivating the theory.
  
  
\end{abstract}

\maketitle




\tableofcontents



\section{Introduction}
  Sheaf cohomology is an algebraic invariant that is largely applied to studying when locally solvable problems fail to be globally solvable. In this paper we will outline the main tools and ideas needed to define sheaf cohmology staying away from most technical details. We then look at an explicit example application of how sheaf cohomolgy can be used to show when a locally sovable problem is or is not globally solvable. This paper is not meant to give a thorough understanding of sheaf cohomology, but rather to motivate it for the reader and point them in a direction where they can investigate further.


\section{Presheaves}
  For a fixed topological space $X,$ we have a category where the objects are the open sets in $X$ and the morphisms between objects are just the inclusion maps. This can easily be seen to be a category since it can be viewed as a subcategory of 
  \textbf{Top} where we restrict the class of topological spaces to be just the members of the topology on $X$ (in the subspace topology) and we restrict the morphisms to be just the inclusion maps. We will denote this category $\textbf{Top}(X).$ \cite{hartshorne_1977} 
  
  \begin{defn}
    A \textit{presheaf} on a topological space $X$ is a contravariant functor 
    \[\mathscr{F}: \textbf{Top}(X) \to \textbf{C}\]
    for some category $\textbf{C}.$
  \end{defn}
   We call $\mathscr{F}$ a presheaf of objects in \textbf{C}. \cite{hartshorne_1977} We will be most interested in presheafs of algeberaic structures, so for instance a presheaf of abelian groups on $X$ would be a presheaf $\mathscr{F}:\textbf{Top}(X) \to \textbf{Ab}$. We typically call the elements of $\mathscr{F}(U)$ the sections of $\mathscr{F}$ of $U.$ Additionally, we denote $\mathscr{F}(U \hookrightarrow V)(s) = s|_U$ for $s \in \mathscr{F}(V)$ call $\mathscr{F}(U \hookrightarrow V)$ the restriciton homomorphism. This notation mirrors a major motivator for the definition of a presheaf which is the presheaf (and in fact sheaf) that associates to each open set of $U \subset \mathbb{R}$ the vector space of continuous real-valued functions on $U.$ 

   \begin{exmp}
    Let $\mathscr{F}$ be the presheaf that associates to each open set $U \subset \mathbb{R}$ the $\mathbb{R}$-vector space of continuous real-valued functions on $U$ and to each inclusion $U \hookrightarrow V$ the restriction map that takes $f: V \to \mathbb{R}$ and maps it to $f|_U: U \to \mathbb{R}.$ We want to check that this is actually a functor in order to know that it is indeed a presheaf. First, we need to know that the restriction homomorphisms are actually linear transformations. But this is clear since if $f,g:V \to \mathbb{R}$ with $U \subset V$ open and $\gamma_1, \gamma_2 \in \mathbb{R},$ then we have 
    \begin{equation*}
      \begin{split}
        (\gamma_1 f + \gamma_2 g)|_U(x) 
        & = (\gamma_1 f + \gamma_2 g)(x) \\
        & = \gamma_1 f(x) + \gamma_2 g(x) \\
        & = \gamma_1 f|_U(x) + \gamma_2 g|_U(x)
      \end{split}
    \end{equation*}
    for all $x \in U,$ so that $(\gamma_1 f + \gamma_2 g)|_U = \gamma_1 f|_U + \gamma_2 g|_U$ and so the restriciton is a linear transformation of vector spaces.
    $\mathscr{F}$ maps the idenitity morphism to the identity morphism since for $f: V \to \mathbb{R}$ we have $f|_V = f.$ Finally, $\mathscr{F}$ respects compositions since if $U \subset V \subset W$ then for $f: W \to \mathbb{R}$ we have $f|_U = (f|_V)|_U.$ Therefore, $\mathscr{F}$ is a functor and so a presheaf of vector spaces on $\mathbb{R}.$
   \end{exmp}



\section{Sheaves}
  A sheaf can be thought of in a few equivalent ways but the most streamlined is to think of a sheaf as just a presheaf that satisfies two additional conditions. In order for this definitions to be coherent from here we will assume that the members of \text{Obj}(\textbf{C}) are some manner of set (e.g. sets, abelian groups, vector spaces, etc.) so that we can talk about inclusion.

  \begin{defn}
    A presheaf $\mathscr{F}$ on $X$ of objects in \textbf{C} is called a \textit{monopresheaf} if given $U \subset X$ such that $U = \bigcup_{\alpha} U_\alpha$ with each $U_\alpha$ open in $X$, and $s,t \in \mathscr{F}(U)$ (hence requirement that the objects in \textbf{C} are some form of set) are such that \[s|_{U_\alpha} = t|_{U_\alpha}\] for all $\alpha$ then $s = t.$ \cite{bredon_1997}
  \end{defn}

  \begin{defn}
    A presheaf is called \textit{conjunctive} if given a collection $U_\alpha$ of open sets in $X,$ and $s_\alpha \in \mathscr{F}(U_\alpha)$ are given such that 
    \[s_\alpha|_{U_\alpha \cap U_\beta}= s_\beta|_{U_\alpha \cap U_\beta}\] for all $\alpha, \beta$ then there exists an element $s \in \mathscr{F}(\bigcup_\alpha U_\alpha)$ such that 
    \[s|_\alpha = s_\alpha\]
    for all $\alpha.$ \cite{bredon_1997}
  \end{defn}
  
  \begin{defn}
    A presheaf is called a \textit{sheaf} if it is a cojunctive monopresheaf. \cite{wiki_Sheaf}
  \end{defn}

  \begin{exmp}
    We will show that the presheaf $\mathscr{F}$ given in Example 2.2 is in fact a sheaf by showing that it is a conjuctive monopresheaf. In the context of $\mathscr{F}$ the requirement to be a monopresheaf amounts to the fact that functions that agree everywhere on their domain are identical, which is clearly true in our case. Being conjuctive amounts to the fact that we can glue continuous functions  on open sets that agree on intersections together to get a continuous function. Therefore, $\mathscr{F}$ is a conjuctive monopresheaf, and so a sheaf.
  \end{exmp}

  


\section{The Category of Sheaves}
   Now, in order to define a category of sheaves on a fixed topological space, we need a notion of morphism between sheaves and that is given simply by the notion of natural transformation.

  \begin{defn}
    Given two presheaves $\mathscr{F}$ and $\mathscr{G}$ of objects in \textbf{C} on a fixed space topological space $X,$ a \textit{morphism} between $\mathscr{F}$ and $\mathscr{G}$ is a natrual transformation from $\mathscr{F}$ to $\mathscr{G}.$ That is a morphism $\eta$ between these two presheaves assocaites to each open set $U$ of $X$ a map $\eta_U: \mathscr{F}(U) \to \mathscr{G}(U)$ such that for any inclusion $U \hookrightarrow V$ we have the following commutative diagram 
    % https://q.uiver.app/?q=WzAsNCxbMCwwLCJcXG1hdGhzY3J7Rn0oVikiXSxbMiwwLCJcXG1hdGhzY3J7R30oVikiXSxbMCwyLCJcXG1hdGhzY3J7Rn0oVSkiXSxbMiwyLCJcXG1hdGhzY3J7R30oVSkiXSxbMCwyLCJcXG1hdGhzY3J7Rn0oVSBcXGhvb2tyaWdodGFycm93IFYpIiwyXSxbMCwxLCJcXGV0YV9WIl0sWzEsMywiXFxtYXRoc2Nye0d9KFUgXFxob29rcmlnaHRhcnJvdyBWKSJdLFsyLDMsIlxcZXRhX1UiLDJdXQ==
    \[\begin{tikzcd}
      {\mathscr{F}(V)} && {\mathscr{G}(V)} \\
      \\
      {\mathscr{F}(U)} && {\mathscr{G}(U)}
      \arrow["{\mathscr{F}(U \hookrightarrow V)}"', from=1-1, to=3-1]
      \arrow["{\eta_V}", from=1-1, to=1-3]
      \arrow["{\mathscr{G}(U \hookrightarrow V)}", from=1-3, to=3-3]
      \arrow["{\eta_U}"', from=3-1, to=3-3]
    \end{tikzcd}\]
    A morphism between sheaves is defined identically. An \textit{isomorphism} of sheaves is just a sheaf morphism with a two-sided inverse. \cite{hartshorne_1977}
  \end{defn}

  Now, we will build up some machinery in order to show that morphisms of sheaves are ``local." That is, they are only concerned with preserving local structure. Most importantly, we will show that isomorphisms between sheaves are those that behave ismorphically locally.

  \begin{defn}
    Suppose we are given a set $(A_i)_{i \in I}$ where each $A_i$ belongs to a fixe algebraic category and $I$ is a directed partially order set. Suppose additionally that we have homomoprhisms $f_{ij}: A_i \to A_j$ exist for all $i \leq j$ in $I$ with the properties that $f_{ii}$ is the idenitity map and $f_{ik} = f_{jk} \circ f_{ij}$ for all $i \leq j \leq k.$ Then the \textit{direct limit} is defined to be the quotient 
    \[
      \lim_{\longrightarrow} (A_i)_{i \in I} = \bigsqcup_i A_i \text{\textfractionsolidus} \sim
    \]
    where $a_i \sim a_j$ if and only if there exists $k \in I$ such that $i,j \leq k$ and $f_{ik}(a_i) = f_{jk}(a_j).$ \cite{direct_limit}
  \end{defn}

  We can see that the requirements to form a direct limit match the situation we have for sheaves perfectly. We just replace $(A_i)_{i \in I}$ with $(\mathscr{F}(U))_{U \in \mathscr{U}}$ for some fixed $x \in X,$ where $\mathscr{U}$ is the set of open sets containing $x$ ordered by set inclusion, and the homomorphisms are just the restriction homomorphisms. Therefore, we can define a stalk as follows.

  \begin{defn}
    The \textit{stalk} of the presheaf $\mathscr{F}$ on $X$ at $x \in X,$ is defined to be 
    \[\mathscr{F}_x = \lim_{\longrightarrow}(\mathscr{F}(U))_{U \in \mathscr{U}}.\]
  \end{defn}

  Now, we give a proposition which formalizes our notion that ``sheaf isomorphisms are local.'' A morphism of sheaves $\eta: \mathscr{F} \to \mathscr{G}$ on topological space $X$ gives in a natural way an induced homomorphism $\eta_x: \mathscr{F}_x \to \mathscr{G}_x$ on each stalk of $\mathscr{F}.$

  \begin{prop}
    A morphism of sheaves $\eta: \mathscr{F} \to \mathscr{G}$ on topological space $X$ is an isomorphism if and only if the induced map $\eta_x: \mathscr{F}_x \to \mathscr{G}_x$ is an isomorphism for each $x \in X.$ \cite{hartshorne_1977}
  \end{prop}

  We now fix our attention solely on sheaves of abelian groups. Using the notion of sheaf morphism we can define a category of sheaves of abelian groups on a fixed topological space $X$ denoted $\textbf{Sh}(X).$ Along with this category we consider an import functor 

  \begin{defn}
    The \textit{global sections functor} is the functor 
    \[\Gamma(X,\underline{\hspace{.25cm}}): \textbf{Sh}(X) \to \textbf{Ab}\]
    that takes a sheaf $\mathscr{F}$ on $X$ and returns the abelian group $\mathscr{F}(X) = \Gamma(X,\mathscr{F}).$ 
  \end{defn}

  This functor will be central to how we define sheaf cohomology. Intuitively this functor drops local information and only keeps what is global, which is why it is germain to the the stated prupose of sheaf cohomology theory of studying how problems that are solvable locally fail to be solvable globally.



\section{Defining Sheaf Cohomology}
  It turns out that the category $\textbf{Sh}(X)$ is similar to the categories $\textbf{Ab}$ and $R-\textbf{Mod}$ in that you are able to effectively ``do homological algebra" in the category. What this means is that we have standard constructions (with some diffuclties that are remedied by defining the sheafification of a presheaf) of subsheaves, quotient sheaves, sheaf kernels, sheaf cokernels, etc. Importantly though we also have notions of cochain complexes of sheaves and short exact sequences. \cite{cuboulder} Before defining sheaf cohomology though we need a few more notions. First is the idea of an ``injective''

  \begin{defn}
    An object $Q$ is said to be an \textit{injective} in a category \textbf{C} if for every morphism $f: X \to Q$ and every monomorphism (the categorical equivalent of an injective map) $g: X \to Y,$ there exists a morphism $h: Y \to Q$ such that the diagram
    % https://q.uiver.app/?q=WzAsMyxbMCwwLCJYIl0sWzAsMSwiUSJdLFsxLDAsIlkiXSxbMCwyLCJmIl0sWzAsMSwiZyIsMl0sWzIsMSwiaCIsMCx7InN0eWxlIjp7ImJvZHkiOnsibmFtZSI6ImRhc2hlZCJ9fX1dXQ==
    \[\begin{tikzcd}
      X & Y \\
      Q
      \arrow["f", from=1-1, to=1-2]
      \arrow["g"', from=1-1, to=2-1]
      \arrow["h", dashed, from=1-2, to=2-1]
    \end{tikzcd}\]
    commutes.
  \end{defn}

  \begin{defn}
    An \textit{injective resolution} of an object $\mathscr{F}$ is an exact sequence
    \[0 \xrightarrow{\hspace{.2cm} \hspace{.2cm}} \mathscr{F} \xrightarrow{\hspace{.2cm} \varepsilon \hspace{.2cm}}I^0 \xrightarrow{\hspace{.2cm} d^0 \hspace{.2cm}} I^1 \xrightarrow{\hspace{.2cm} d^1 \hspace{.2cm}} I^2 \xrightarrow{\hspace{.2cm} d^2 \hspace{.2cm}} \cdots \] 
    with each $I^n$ an injective.
  \end{defn}

  An important result is that $\textbf{Sh}(X)$ ``has enough injectives'' which is a techinical term that amounts to the fact that for any object we can always find such an injective resolution in $\textbf{Sh}(X).$ Now, given an injective resolution in $\textbf{Sh}(X)$, what happens when we apply the global section functor to it? That is, for the injective resolution 
  \[0 \xrightarrow{\hspace{.2cm} \hspace{.2cm}} \mathscr{F} \xrightarrow{\hspace{.2cm} \varepsilon \hspace{.2cm}}I^0 \xrightarrow{\hspace{.2cm} d^0 \hspace{.2cm}} I^1 \xrightarrow{\hspace{.2cm} d^1 \hspace{.2cm}} I^2 \xrightarrow{\hspace{.2cm} d^2 \hspace{.2cm}} \cdots \] 
  what can we say about the sequence 
  \[0  \xrightarrow{\hspace{.2cm} \hspace{.2cm}} \Gamma(X,I^0) \xrightarrow{\hspace{.2cm} \Gamma(d^0) \hspace{.2cm}} \Gamma(X,I^1) \xrightarrow{\hspace{.2cm} \Gamma(d^1) \hspace{.2cm}} \Gamma(X,I^2) \xrightarrow{\hspace{.2cm} \Gamma(d^2) \hspace{.2cm}} \cdots\] 

  This sequence is not in general exact, but it is a cochain complex and so we can consider the $n$-th cohomology groups which we denote 
  \[R^n \mathscr{F}(X) = \ker \Gamma(d^n) / \text{im} \Gamma(d^{n-1})\]
  and further we have that $R^0 \mathscr{F} = \mathscr{F}(X).$ $R^n\mathscr{F}(X)$ is called the \textit{$n$-th right derived functor} of $\mathscr{F}.$
  
  It turns out that the right derived functors of a space are independent of the injective resolution chosen, and so we can just talk about the right derived functors with no concern about them being well-defined. \cite{cuboulder} Now, we define sheaf cohomology on a topological space $X$ using this construction.

  \begin{defn}
    Given a space $X$ and a sheaf $\mathscr{F}$ on $X,$ we define a the $n$-th sheaf cohomology group of $\mathscr{F}$ to be 
    \[H^n(X,\mathscr{F}) = R^n\mathscr{F}.\]
  \end{defn}


\section{Example of an Application of Sheaf Cohomology}
  We now bring our attention to how we can use sheaf cohomology to investigate when local properties fail to become global. 

  Consider a complex manifold $X.$ We will define three seperate sheaves on $X,$ namely $\underline{\mathbb{Z}},$ $\mathscr{O}_X,$ and $\mathscr{O}_X^*$ but avoiding showing in detail that each is in fact a sheaf. 
  
  The sheaf $\underline{\mathbb{Z}}$ is called the \textit{constant sheaf}, and is the sheaf on $X$ that sends an open set $U \subset X$ to $\underline{\mathbb{Z}}(U)$ which is the set of continuous functions from $U$ to $\mathbb{Z}$ in the discrete topology. Despite its name, not all sections in $\underline{\mathbb{Z}}(U)$ are constant, such a presheaf would not actually be a sheaf. 

  The sheaf $\mathscr{O}_X$ is the sheaf of holomorphic functions. That is, the sheaf that sends each open set in $X$ to the set of holomorphic functions on $U.$ Similiarly defined, $\mathscr{O}_X^*$ is the sheaf of nonvanishing holomorphic functions.

  We have a sequence of sheaves on $X$ given by 
  \[0 \xrightarrow{ \hspace{.25cm}  \hspace{.25cm}}  \underline{\mathbb{Z}} \xrightarrow{ \hspace{.25cm} \iota \hspace{.25cm}}  \mathscr{O}_X \xrightarrow{ \hspace{.2cm} exp \hspace{.2cm}} \mathscr{O}_X^* \xrightarrow{ \hspace{.25cm}  \hspace{.25cm}} 0\]
  which turns out to be exact. \cite{cuboulder}

  Now, similarly to a short exact sequence of chain complexes giving rise to a long exact sequence in homology, a short exact sequence of sheaves gives rise to long exact sequence of derived functors. \cite{derived_functor} In our case, that means we have a long exact sequence 
  \[
  0 
    \xrightarrow{ \hspace{.1cm}  \hspace{.1cm}}  
  \Gamma(X,\underline{\mathbb{Z}}) 
    \xrightarrow{ \hspace{.1cm}  \hspace{.1cm}}  
  \Gamma(X,\mathscr{O}_X) 
    \xrightarrow{ \hspace{.1cm} exp \hspace{.1cm}} 
  \Gamma(X,\mathscr{O}_X^*) 
  \xrightarrow{ \hspace{.1cm}  \hspace{.1cm}} 
    H^1(X,\underline{\mathbb{Z}})
  \xrightarrow{ \hspace{.1cm}  \hspace{.1cm}}
    H^1(X,\mathscr{O}_X)
  \xrightarrow{ \hspace{.1cm}  \hspace{.1cm}}
  \cdots
  \]

  Now, from this exact sequence we can deduce quite a bit of information by recognizing that $\Gamma(X,\mathscr{O}_X)$ is the set of holomorphic functions defined on all of $X.$ Since the sequence is exact the map 
  \[\Gamma(X,\mathscr{O}_X) 
  \xrightarrow{ \hspace{.1cm} exp \hspace{.1cm}} 
  \Gamma(X,\mathscr{O}_X^*)\] 
  is surjective if and only if $H^1(X,\underline{\mathbb{Z}}) = 0.$ This shows that even though every nonvanishing function $g \in \mathscr{O}_X^*$ is locally representable as $g = e^f$ for some $f \in \mathscr{O}_X,$ this same problem is only solvable globally if $H^1(X,\underline{\mathbb{Z}}) = 0.$ In fact, $H^1(X,\underline{\mathbb{Z}})$ is readily computabile since for appropraite spaces we actually have $H^i(X,\underline{\mathbb{Z}})$ is equivalent to the the singular cohomology group, that is $H^i(X,\underline{\mathbb{Z}}) \cong H_{sing}^i(X, \mathbb{Z}).$ \cite{cuboulder} This problem encapsulates how sheaf cohomology can be used to study how locally solvable problems can fail to be globally solvable.




\bibliographystyle{unsrt}
\bibliography{bib}
\nocite{wiki_Sheaf_Cohomology}

\end{document}
